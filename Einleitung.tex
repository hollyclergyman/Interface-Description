% !TEX root = Ausarbeitung.tex

\chapter{Einleitung}
Als integriertes System biologischer und physikalischer Faktoren ist die Tierhaltung auf die klare Definition von Zielsetzungen und die beständige Anpassung des Managements an die Zielgrößen angewiesen \autocites[3]{wathes_c._m._is_2008}[2]{klindworth_m._prozesssteuerung_2006}. Die Rekombination, Verdichtung und Modellierung der verfügbaren Daten zu Informationen für die Anwendung im realen Prozess der Tierhaltung ist dabei, in herkömmlichen Systemen, eine Aufgabe des Tierhalters \autocites[2 ff]{wathes_c._m._is_2008}[2]{ klindworth_m._prozesssteuerung_2006}. % \autocites[]{}[]{} enables multiple sources for one citation
Im Rahmen des PLF lässt sich diese Aufgabe mittels Computer-basierter Verfahren in Echtzeit automatisieren, das Management der Tiere individualisieren und genauere Vorhersagen über zukünftige Ereignisse treffen \autocites[2 ff]{wathes_c._m._is_2008}[33 ff]{gallmann_e._methoden_2016}.\\
Durch die Entwicklung neuer Sensoren, deren Anwendung im praktischen Einsatz der Haltung von Milchkühen und Verbesserungen bei der zugehörigen Computertechnik, lassen sich mittlerweile wesentlich mehr Indikatoren über den Zustand und die Gesundheit von Milchkühen sammeln und auswerten \autocite[5]{schulze_c._hybride_2008}. Auf Grundlage mathematischer Modelle lassen sich aus den Daten einzelner Sensoren Aussagen über die Zielgrößen des Verfahrens treffen \autocite[4]{wathes_c._m._is_2008}. Zugleich können, durch die gestiegene Leistung von Hard- und Software, Gleichungen mit mehr Variablen aufgelöst werden und biologischen Faktoren und (zeitabhängige) Unterschiede der Tiere, durch kontinuierliche Erfassung, besser abgebildet werden \autocite[657 ff]{aerts_j.-m._modelling_2010}.\\
Da in der Haltung von Milchkühen der Fokus in hohem Maße auf dem Einzeltier liegt, stellt die Einzeltiererkennung die Schlüsseltechnik für den Einsatz weiterer Erkennungs- und Erfassungstechnologien dar \autocite[6]{klindworth_m._prozesssteuerung_2006}. 
Für den Tierhalter ergibt sich somit die Möglichkeit, die Kontrolle und das Management der Tiere zu optimieren, Kosten zu sparen und das Tierwohl zu verbessern \autocite[2]{banhazi_t._m._precision_2012}. Zugleich lassen sich die externen Effekte der Tierhaltung, im Bezug auf die Umwelt minimieren \autocite[3]{wathes_c._m._is_2008}.\\
Für die Vorbereitung, Unterstützung und Durchführung von Entscheidungen und die Ausgabe von weitergehenden Informationen ist die Verknüpfung von unterschiedlichen Sensorsystemen und Datenherkünften notwendig \autocite[22]{gallmann_e._methoden_2016}. Aus dieser Anforderung ergibt sich die Integration unterschiedlicher Daten, um den einzelnen Tieren, wie auch der gesamten Gruppe, konsistente Daten zuordnen zu können \autocite[22]{ammon_c._probleme_2006}.

%\chapter{Material und Methoden}