% !TEX root = Ausarbeitung.tex
\chapter*{Abkürzungen}
\addcontentsline{toc}{chapter}{Abkürzungen}
\chaptermark{Abkürzungen} % wichtig, damit nicht erwähnte Abschnitte im Header auftauchen
\begin{acronym}
 \acro{ACID}{atomicity, consistency, integrity, durability}
 \acro{ADED}{Agricultural Data Exchange Dictionary}
 \acro{ADIS}{Agricultural Data Interchange Syntax}
 \acro{ama}{Agrarmarkt Austria, Wien}
 \acro{aws}{Amazon Web Service}
 \acro{BHV1}{Bovines Herpesvirus 1}
 \acro{BTV}{Blauzungenvirus}
 \acro{BVD}{Bovine Virusdiarrhoe}
 \acro{DDI}{Data Dictionary, ADED-Datenelement in ADIS}
 \acro{DIN}{Deutsches Institut für Normung e.V.}
 \acro{EG}{Europäische Gemeinschaft, hier im Kontext legislativer Maßnahmen der EU}
 \acro{EU}{Europäische Union}
 \acro{Hi-Tier}{Herkunftssicherungs- und Informationssystem für Tier}
 \acro{HTML}{Hypertext Markup Language}
 \acro{ISO}{International Organization for Standardization}
 \acro{json}{JavaScript Online Notation}
 \acro{LKV}{Landeskontollverband}
 \acro{MLP}{Milchleistungsprüfung}
 \acro{PLF}{Precision Livestock Farming}
 \acro{RDV}{Rinderdatenverbund, RDV EDV Entwicklungs- und Vertriebs GmbH, München}
 \acro{REST}{Representational State Transfer}
 \acro{rzv}{Rechenzentrum Verden}
 \acro{SQL}{Stuctured Query Language}
 \acro{URL}{Uniform Resource Locator}
 \acro{vit w.V.}{Vereinigte Informationssysteme Tierhaltung w. V., Verden/Aller}
 \acro{xml}{Extensible Markup Language}
\end{acronym}
